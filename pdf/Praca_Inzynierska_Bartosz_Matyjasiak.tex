\documentclass{SGGW-thesis}
\title{TODO}
\Etitle{TODO}
\author{Bartosz Matyjasiak}
\date{2019\footnote{Dokument skompilowano z klasą {\tt SGGW-thesis} w wersji 1.05. Aktualną wersję klasy można pobrać ze strony \url{http://stud.lchmiel.pl} $\rightarrow$ Seminarium dyplomowe.}}
\university{Szkoła Główna Gospodarstwa Wiejskiego\\w Warszawie}
\dep{Wydział Zastosowań Informatyki i Matematyki}
\album{185117}
\thesis{Praca dyplomowa inżynierska}
\course{Informatyka}
\promotor{dr.\ hab.\ inż.\ imie nazwisko, prof.\ SGGW} %TODO
\pworkplace{Wydział Zastosowań Informatyki i Matematyki\\Katedra Informatyki}

\usepackage{hyperref}

\begin{document}
\maketitle
\statementpage
\abstractpage
{TODO}
{TODO}
{TODO, TODO, TODO}
{TODO}
{TODO}
{TODO, TODO, TODO}


{
  % Spis treści może być złożony z pojedynczą interlinią, np. jeśli jedna linia wychodzi na następną stronę.
  % W przeciwnym razie spis treści wstawić bez powyższego rozkazu i klamry.
  \doublespacing
  \tableofcontents
}

\startchapterfromoddpage % niezależnie od długości spisu treści pierwszy rozdział zacznie się na nieparzystej stronie


\chapter{Wstęp}


\section{Cel i zakres pracy}


\chapter{Użycie}


\section{Instalacja}


\chapter{Podsumowanie i wnioski}


\begin{thebibliography}{9}

\end{thebibliography}

\beforelastpage

\end{document} 