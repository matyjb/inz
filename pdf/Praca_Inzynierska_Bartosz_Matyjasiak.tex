\documentclass{SGGW-thesis}
\title{Projekt i implementacja aplikacji mobilnej wyświetlającej aktualne lokalizacje autobusów oraz tramwajów w Warszawie}
\Etitle{TODO}
\author{Bartosz Matyjasiak}
\date{2020\footnote{Dokument skompilowano z klasą {\tt SGGW-thesis} w wersji 1.05. Aktualną wersję klasy można pobrać ze strony \url{http://stud.lchmiel.pl} $\rightarrow$ Seminarium dyplomowe.}}
\university{Szkoła Główna Gospodarstwa Wiejskiego\\w Warszawie}
\dep{Wydział Zastosowań Informatyki i Matematyki}
\album{185117}
\thesis{Praca dyplomowa inżynierska}
\course{Informatyka}
\promotor{dr.\ hab.\ inż.\ Leszek Chmielewski, prof.\ SGGW}
\pworkplace{Wydział Zastosowań Informatyki i Matematyki\\Katedra Informatyki} % TODO

\usepackage{hyperref}

\begin{document}
\maketitle
\statementpage
\abstractpage
{TODO}
{TODO}
{TODO, TODO, TODO}
{TODO}
{TODO}
{TODO, TODO, TODO}


{
  % Spis treści może być złożony z pojedynczą interlinią, np. jeśli jedna linia wychodzi na następną stronę.
  % W przeciwnym razie spis treści wstawić bez powyższego rozkazu i klamry.
  \doublespacing
  \tableofcontents
}

\startchapterfromoddpage % niezależnie od długości spisu treści pierwszy rozdział zacznie się na nieparzystej stronie


\chapter{Wstęp}
% po co taka aplikacja
\section{Założenia}
% wymienić pare wymagań dla takiej aplikacji
% motywy
% aktualne pozycje 
% możliwość sprawdzenia rozkładu jazdy na danym przystanku
% dodawanie do ulubionych ptzystanków i linii
\section{Grafiki koncepcyjne}
% wybrane chyba

\chapter{Implementacja}
% tu napisać o plikach api, o wybranym react nativie, o paczce od map do reacta
% napisać o GlobalContext, który trzyma wszystkie ustawienia aplikacji w tym ulubione itp
% w tym też logike ale bez logiki aktualizacji pozycji autobusów ze względu na optymalizacje
% by aplikacja nie przerosywowała się co 10 sekund, a jedynie mapa
\section{Ulubione przystanki}
% nie wiem czy jeszcze to bedzie potrzebne

\section{Radar}
% napisać że zbiera linie z przystanków które są w granicach radaru 
% tu napisać o tym hacku z firebasem (tu napisać ze ze względu na 573925 zapytań o linie jakie są
% na danym przystanku)

\section{Aktualizacja pozycji pojazdów}
% napisać o funkcji API której użyłem do tego celu
% że co 10 sek
% odpytuje wszystkie linie z ulubionych i dodaje do wyniku wszystko
% odpytuje wszystkie linie z radaru, które nie są w ulubionych i dodaje tylko
%   te, które są w zasięgu radaru
\section{Pinezki przystanków}
% pobieranie pliku z firebasem wspomnianego w sekcji Radar
% tu napisać ze odrazu ten plik już ma pogrupowane przystanki po id
%   by zoptymalizować (i o tym ze grupa ma tez wyliczoną średnią pozycje)
% tu opisac logike wyswietlania
\section{Ukrycie kluczy API w kodzie}
% napisać że uzywam dwóch kluczy: do map i do warszawskiego api
% że używam git'a do zarządzania zmianami w kodzie
% i że klucze nalezy ukrywać w plikach np. .env
% potem że żeby dodac klucz map google do androidManifest użyłem pakietu dotenv
% a ze do warsaw api poprzez BuildConfig z paczki react-native-config



\chapter{Podsumowanie i wnioski}


\begin{thebibliography}{9}

\end{thebibliography}

\beforelastpage

\end{document} 