\def\code#1{\texttt{#1}}
\documentclass{SGGW-thesis}
\title{Projekt i~implementacja aplikacji mobilnej wyświetlającej aktualne lokalizacje autobusów oraz tramwajów w~Warszawie}
\Etitle{TODO}
\author{Bartosz Matyjasiak}
\date{2020\footnote{Dokument skompilowano z~klasą {\tt SGGW-thesis} w~wersji 1.05. Aktualną wersję klasy można pobrać ze strony \url{http://stud.lchmiel.pl} $\rightarrow$ Seminarium dyplomowe.}}
\university{Szkoła Główna Gospodarstwa Wiejskiego\\w Warszawie}
\dep{Wydział Zastosowań Informatyki i~Matematyki}
\album{185117}
\thesis{Praca dyplomowa inżynierska}
\course{Informatyka}
\promotor{dr.\ hab.\ inż.\ Leszek Chmielewski, prof.\ SGGW}
\pworkplace{Instytut Informatyki Technicznej\\Katedra Sztucznej Inteligencji}

\usepackage{hyperref}

\begin{document}
\maketitle
\statementpage
\abstractpage
{TODO}
{TODO}
{TODO, TODO, TODO}
{TODO}
{TODO}
{TODO, TODO, TODO}


{
  % Spis treści może być złożony z~pojedynczą interlinią, np. jeśli jedna linia wychodzi na następną stronę.
  % w~przeciwnym razie spis treści wstawić bez powyższego rozkazu i~klamry.
  \doublespacing
  \tableofcontents
}

\startchapterfromoddpage % niezależnie od długości spisu treści pierwszy rozdział zacznie się na nieparzystej stronie


\chapter{Wstęp}
% po co taka aplikacja
W dużych miastach komunikacja miejska jest kluczowym aspektem dla mieszkańców.
Niestety duże miasta, w~tym Warszawa, boryka się z~korkami, wypadkami, robotami drogowymi i~innymi przez
co autobusy czy tramwaje często nie jeżdżą według rozkładu jazdy. Dlatego dobrą informacją dla podróżującego
jest lokalizacja GPS autobusu lub tramwaju. Warszawa udostępnia takie dane w~projekcie "Otwarte Dane" jednak są one
w postaci nieczytelniej dla przeciętnego człowieka. Rozwiązaniem może być aplikacja mobilna,
dzięki której użytkownik będzie widział na mapie kiedy dokładnie przyjedzie autobus lub tramwaj.
% Dzięki temu też będzie wiedział czy powinien się pospieszyć idąc na przystanek.
\section{Założenia}
% wymienić pare wymagań dla takiej aplikacji
% motywy
% aktualne pozycje 
% możliwość sprawdzenia rozkładu jazdy na danym przystanku
% dodawanie do ulubionych ptzystanków i~linii
Aplikacja powinna:
\begin{itemize}
  \item{Pokazywać aktualne pozycje autobusów i~tramwajów na mapie}
  \item{Pokazywać pozycje przystanków na mapie}
  \item{Udostępniać rozkłady jazdy na każdym z~przystanków}
  \item{Umożliwiać na dodanie linii autobusówej lub tranwajowej do ulubionych}
  \item{Umożliwiać na dodanie przystanku do ulubionych}
  \item{Wspierać dwa motywy:}
  \begin{itemize}
    \item{Jasny}
    \item{Ciemny}
  \end{itemize}
\end{itemize}
\section{Grafiki koncepcyjne}
% wybrane chyba

\chapter{Implementacja}
% o~wybranym react nativie, o~paczce od map do reacta
Do implementacji wybrałem React-Native stworzony przez Facebook. Pozwala on na zrobienie aplikacji na telefony
z systemem Android oraz iOS przy pomocy jednego kodu źródłowego napisanego w~języku JavaScript XML
(w skrócie JSX). Wybrałem także moduł react-native-maps, który jest odpowiedzialny za wyświetlanie
mapy Google oraz sterowanie nią.

\section{Publiczne API Warszawy}
% TODO: odnośniki do plików dokumentacji tych endpointów
Miasto udostępnia dane w~postaci publicznego API. Z~pośród wielu punktów końcowych tego API są dostępne:
\begin{itemize}
  \item{Pozycje pojazdów danej linii}
  \item{Zbiór linii, które odjeżdzają z~danego przystanku}
  \item{Rozkład jazdy dla danej linii z~danego przystanku}
  \item{Zbiór wszystkich przystanków}
\end{itemize}
Co ważne pozycje pojazdów są aktualizowane co 10 sekund i~też z~taką częstotliwością będą
aktualizowane w~aplikacji.
Wszystkie z~wymienionych punktów końcowych API zaimplementowałem w~klasie \code{WarsawAPI}.


\section{Komponent GlobalContextProvider}
% TODO: odnośnik do Contextu w~React
Stworzyłem komponent nadrzędny, przy pomocy kontekstu dostępnego w~React, odpowiedzialny za całą logike
aplikacji.
Komponent ten przechowuje zmienne:
\begin{itemize}
  \item{Zbiór wszystkich przystanków}
  \item{Zbiór ulubionych linii}
  \item{Zbiór ulubionych przystanków}
  \item{Aktualny wyświetlany region mapy}
  \item{Pozycje radaru oraz jego promień}
  \item{Zaznaczony przystanek lub pojazd}
\end{itemize}
Oraz funkcje do modyfikacji zmiennych.
Komponent też przechowuje referencje do komponentu mapy oraz udostępnia funkcje
od sterowania nią.

Jednak nie umieściłem w~nim logiki aktualizowania pozycji pojazdów gdyż każda zmiana stanu
tego komponentu powoduje przerysowanie wszystkich komponentów GlobalContext.Consumer, a~wraz nim wszystkich jego dzieci.
Przez to, że ten komponent jest używany w~wielu miejscach to każda aktualizacja pozycji pojazdów, a~ta
jest co 10 sekund, powodowałaby przerysowanie całej aplikacji. To wiązałoby się z~utratą na szybkości działania
aplikacji.

\section{Aktualizacja pozycji pojazdów}
% napisać o~funkcji API której użyłem do tego celu
% że co 10 sek
% odpytuje wszystkie linie z~ulubionych i~dodaje do wyniku wszystko
% odpytuje wszystkie linie z~radaru, które nie są w~ulubionych i~dodaje tylko
%   te, które są w~zasięgu radaru
By aktualizacja przebiegała sprawnie wraz z~wyświetlaniem logike aktualizacji umieściłem w~komponencie GMap.
Jest to komponent, który jako dziecko posiada tylko komponent map. Jest to ważne bo gdy tylko zmieni się stan
komponentu GMap, a~ten będzie się zmieniał co 10 sekund, to wywoła to przerysowanie tylko komponentu map.
W tym komponencie zaimplementowałem funkcje, która:
\begin{enumerate}
  \item{Dla każdej ulubionej lub wykrytej przez radar linii są pobierane pozycje pojazdów tych linii}
  \item{Jako pojazdy do wyświetlenia są brane pod uwage tylko te pojazdy, które są z~linii ulubionej lub w~promieniu radaru oraz czas wysłania sygnału GPS nie jest starszy niż 6 minut}
\end{enumerate}
Funkcja ta jest uruchamiana co minute za pomocą funkcji \code{setTimeout} wbudowanej w~język JavaScript

\section{Pinezki przystanków}
% pobieranie pliku z~firebasa, przygotowanego (tu o~grupach)
% tu napisać ze odrazu ten plik już ma pogrupowane przystanki po id
%   by zoptymalizować (i o~tym ze grupa ma tez wyliczoną średnią pozycje)
% tu opisac logike wyswietlania
\section{Radar}
% napisać że zbiera linie z~przystanków które są^{}granicach radaru 
Głównym celem radaru jest pokazywanie pinezek pojazdów linii z~poza ulubionych. Ogranicza on tez ilość
pinezek do narysowania. Jest to bardzo ważne bo autobusów i~tramwajów w~Warszawie jest zbyt duża
ilość by efektywnie pokazać je wszystkie na raz.

Działanie radaru jest następujące:
\begin{enumerate}
  \item{Użytkownik za pomocą przycisku ustawia pozycje radaru na środku regionu mapy, który jest aktualnie wyświetlany}
  \item{Dla każdego z~grup przystanków jest sprawdzane, czy średnia pozycja grupy jest w~promieniu radaru}
  \item{Jeśli jest to wszystkie linie z~każdego przynkanka danej grupy są dodawane do zbioru linii radaru. Wszystkie elementy tego zbioru są unikalne}
\end{enumerate}
% tu napisać o~tym hacku z~firebasem (tu napisać ze ze względu na 573925 zapytań o~linie jakie są
% na danym przystanku)
\section{Ukrycie kluczy API w~kodzie}
% napisać że uzywam dwóch kluczy: do map i~do warszawskiego api
% że używam git'a do zarządzania zmianami w~kodzie
% i~że klucze nalezy ukrywać w~plikach np. .env
% potem że żeby dodac klucz map google do androidManifest użyłem pakietu dotenv
% a~ze do warsaw api poprzez BuildConfig z~paczki react-native-config



\chapter{Podsumowanie i~wnioski}


\begin{thebibliography}{9}

\end{thebibliography}

\beforelastpage

\end{document} 